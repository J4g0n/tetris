% Created 2013-12-09 Mon 08:03
\documentclass[11pt]{article}
\usepackage[utf8]{inputenc}
\usepackage[T1]{fontenc}
\usepackage{fixltx2e}
\usepackage{graphicx}
\usepackage{longtable}
\usepackage{float}
\usepackage{wrapfig}
\usepackage{rotating}
\usepackage[normalem]{ulem}
\usepackage{amsmath}
\usepackage{textcomp}
\usepackage{marvosym}
\usepackage{wasysym}
\usepackage{amssymb}
\usepackage{hyperref}
\tolerance=1000
\author{Simon Andreux, Julien Gamba, Clément Schreiner}
\date{2013-12-08}
\title{Implémentation en scala d'un tetris, avec intelligence artificielle}
\hypersetup{
  pdfkeywords={},
  pdfsubject={},
  pdfcreator={Emacs 24.3.1 (Org mode 8.2.1)}}
\begin{document}

\maketitle
\tableofcontents



\section{Introduction - Pourquoi Scala ?}
\label{sec-1}

Nous n'aimons pas Java, notamment pour la verbosité du code.

Clément, ayant un peu d'expérience en programmation fonctionnelle, a
proposé scala, qui a l'avantage d'être entièrement objet, mais aussi
de proposer des fonctionnalités qu'on trouve généralement dans les
langages fonctionnels tels que le pattern matching et l'inférence de
type.

Il est de plus entièrement intéropérable avec Java, peut importer directement ses bibliothèques, et se compile en bytecode exécutable par la JVM.
\section{Modélisation, structures de données}
\label{sec-2}

\section{Interface réseau}
\label{sec-3}
Pour assurer la communication entre l'IA et la partie graphique du
projet, nous avons choisi de passer par un serveur TCP.

Scala propose plusieurs bibliothèques. Nous avons retenu \texttt{Netty}
pour ce projet. Elle est assez standard, et est donc très bien documentée.
Elle utlise des bibliothèques Java, ce qui impose d'utiliser des
méthodes Java directement dans le code Scala.

En soi, ce n'est pas très contraignant, mais cela peut poser certains
problèmes dans certains cas. Par exemple, dans notre programme, un fois
le serveur lancé, il est possible pour l'utilisateur -- et plus
particulièrement pour l'IA -- d'envoyer des messages au serveur
directement sous forme de \texttt{String}. Ces \texttt{String}s 
passent par un buffer (un \texttt{BigEndianHeapChannelBuffer}), 
qui encode les \texttt{String}s et les rend illisibles sans 
traitement préalable. Cela impose l'utilisation d'un handler qui va agir au
niveau de la \texttt{PipelineFactory} pour décoder les flux et les rendre 
utilisables.

\section{Problèmes rencontrés}
\label{sec-4}

\subsection{Récupérer une ligne dans la matrice de la grille}
\label{sec-4-1}

L'ensemble des blocs de la grille (excepté la pièce courante, qui n'est pas fixe) est représenté par une matrice, implémentée sous forme d'Array d'Array.

Ainsi, pour une matrice de 3x3:

\begin{verbatim}
scala> import Array.ofDim
scala> var blocks = ofDim[Boolean](3, 3)
blocks: Array[Array[Boolean]] = Array(Array(false, false, false),
  Array(false, false, false),
  Array(false, false, false))
scala> val b = blocks(0)(2)
b: Boolean = true
\end{verbatim}

\emph{b} correspond ici au bloc d'abscisse 0 et d'ordonnée 2, et \emph{blocks(0)} à la première colonne

Pour vérifier la présence de lignes complètes, on a besoin de récupérer une ligne. Nous avons trouvé une solution fonctionnelle et élégante utilisant \emph{map}:

\begin{verbatim}
def getLine(blocks) = blocks.map(_(y))
\end{verbatim}
\subsection{Définir simplement les formes des pièces}
\label{sec-4-2}

Après plusieurs tentatives plus moins élégantes, nous avons finalement
choisi une solution assez simple. Chaque forme est définie par un
objet implémentant le trait ShapeKind, qui comport deux données: la
couleur, et une liste de coordonnées relatives des blocs la composant,
pour chaque orientation possible. Concrètement: une liste de liste de
paires d'entiers.

\section{Organisation}
\label{sec-5}

Nous avons réparti le travail ainsi :

\begin{itemize}
\item Simon : intelligence artificielle
\item Julien : interface réseau entre la GUI et l'AI
\item Clément : interface graphique
\item Simon et Clément : structures de données représentant le jeu
\end{itemize}

Nous avons utilisé git pour faciliter la collaboration.
% Emacs 24.3.1 (Org mode 8.2.1)
\end{document}
